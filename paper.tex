% !TeX spellcheck=de_DE
% !TeX program=xelatex
\documentclass[12pt]{article}
\PassOptionsToPackage{hyphens}{url}
\usepackage{newclude}
\usepackage{hyperref}
\usepackage[babelshorthands]{polyglossia}
\usepackage{fontspec}
\usepackage{array}
\usepackage{amsmath, amsfonts, amssymb}
\usepackage[a4paper, hmargin=2.5cm, vmargin=2.0cm]{geometry}
\usepackage{enumerate}
\usepackage{xspace}
\usepackage{graphicx}
\usepackage{xcolor}
\usepackage{enumerate}
\usepackage{fontawesome}
\usepackage{subfigure}
\usepackage{wrapfig}
\usepackage{todonotes}
\usepackage{titlesec}
\usepackage{scrpage2}

\setmainlanguage{german}

\makeatletter
\def\email#1{{\tt#1}}
\def\subtitle#1{\gdef\@subtitle{#1}}
\def\institute#1{\gdef\@institute{#1}}
\def\authors#1#2{\author{#2}}
\makeatother
\title{Omega Automaten und LTL}
\subtitle{Seminararbeit zu der Veranstaltung \\Formale Grundlagen der Informatik 3}
\authors{Jennifer~Soltau \& Felix~Ortmann}{Jennifer~Soltau und Felix~Ortmann\\ \email{\{2soltau,0ortmann\}@informatik.uni-hamburg.de}}
\institute{Universität Hamburg\\Fachbereich Informatik\\Vogt-Kölln-Straße 30\\22527 Hamburg}



\newcolumntype{x}[1]{>{\centering\arraybackslash}m{#1}}

\def\calibri#1{\begingroup\fontspec{Calibri}\selectfont#1\endgroup}

\newcommand{\subfigureautorefname}{Abb.}
\renewcommand{\figureautorefname}{Abbildung}
\renewcommand{\sectionautorefname}{Abschnitt}
\renewcommand{\subsectionautorefname}{Unterabschnitt}

\newtheorem{satz}{Satz}[section]
\newtheorem{defi}{Definition}[section]

\linespread{1.1}

\pagestyle{scrheadings}
\clearscrheadings
\ofoot{\pagemark}

\titleformat{\section}[hang]{\fontsize{16pt}{16pt}\selectfont\bf}{\thesection\quad}{0pt}{}{}
\titlespacing*{\section}{0pt}{24pt}{6pt}
\titleformat{\subsection}[hang]{\fontsize{14pt}{14pt}\selectfont\bf}{\thesubsection\quad}{0pt}{}{}
\titlespacing*{\subsection}{0pt}{18pt}{6pt}
\titleformat{\subsubsection}[hang]{\fontsize{12pt}{12pt}\selectfont\bf}{\thesubsubsection\quad}{0pt}{}{}
\titlespacing*{\subsubsection}{0pt}{12pt}{6pt}

\begin{document}
\urlstyle{same}
	
\makeatletter
\def\@maketitle{%
	\newpage
	\null
	\vskip 2em%
	\begin{center}%
		\let \footnote \thanks
		{\LARGE\bfseries \@title \par}%
		\vskip .5em%
		{\Large \@subtitle \par}%
		\vskip 1.5em%
		{\large
			\lineskip .5em%
			\begin{tabular}[t]{c}%
				\@author
			\end{tabular}\par}%
		\vskip 1em%
		{\large \@date}%
	\end{center}%
	\par
	\vskip 1.5em}
\makeatother

\maketitle
\begin{abstract}
	% Motivation bzw. Zusammenfassung
	Der hier vorliegende Aufsatz stellt kurz die Grundeigenschaften von Omega Automaten (insbesondere Büchi Automaten) sowie von LTL Formeln vor. Ziel und Kernpunkt der Arbeit ist, eine Verbindung zwischen (alternierenden) Büchi Automaten und LTL Formeln herzustellen, um anschließend jegliche Eigenschaftsbeweise von LTL Formeln mit automatentheoretischen Mitteln bewältigen zu können. Wir richten uns nach Moshe Y. Vardi und dessen Arbeit "`An automate-theoretic approach to linear temporal logic"' \cite{vardi+96}. Zusammenfassend werden Vorteile dieses Ansatzes in Bezug auf Modelchecking diskutiert.
	
\end{abstract}

\include*{tex/einleitung}
\include*{tex/omega-automaten}
\include*{tex/ltl}
\include*{tex/aba-zu-ltl}
\include*{tex/ba-zu-ltl}
\include*{tex/modelchecking}
\include*{tex/zusammenfassung}


\nocite{*}
\bibliographystyle{abbrv}
\bibliography{bib}
\end{document}

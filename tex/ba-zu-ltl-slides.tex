\begin{frame}{LTL auf Büchi-Automaten}
\begin{block}{Schritte:}
\begin{itemize}
\item Überführung von alternierenden Büchi-Automaten zu normalen nichtdeterministischen Büchi-Automaten
\item Schlussfolgerung für LTL mit Büchi-Automaten
\end{itemize}
\end{block}
\end{frame}

\begin{frame}{Alternierende Büchi-Automaten zu Büchi-Automaten}
\begin{block}{Satz}
Sei A ein alternierender Büchi-Automat. Dann gibt es einen nicht nichtdeterministischen Büchi-Automaten $A_n$ sodass $L_{\omega}(A_n)=L_{\omega}(A)$ gilt.
\end{block}
\end{frame}

\begin{frame}{Alternierende Büchi-Automaten zu Büchi-Automaten}
\begin{block}{Beweis}
\begin{itemize}
\item $A_n$ nimmt Lauf von A an und merkt sich gesamte Ebene eines Laufes
\item einzelnen Ebenen eines Laufes sind Bäume
\item $A_n$ merkt sich das Auftreten von akzeptierenden Zuständen
\item Aufteilung der Ebene eines Laufs in zwei Mengen
\begin{itemize}
\item Zweige, die vor kurzem einen Endzustand erreicht haben
\item Zweige, die vor kurzem keinen Endzustand erreicht haben
\end{itemize}
\end{itemize}
\end{block}
\end{frame}

\begin{frame}{Alternierende Büchi-Automaten zu Büchi-Automaten}
\begin{block}{Beweis (Konstruktion)}
Gegeben $A=(\Sigma,S, s^0, \rho, F)$. Dann $A_n=(\Sigma,S_n,S^0,\rho_n,F_n)$ mit
\begin{itemize}
\item $S_n=2^S\times 2^S$
\item $F_n=\emptyset\times 2^S$
\item und für die Transitionsfunktion gilt: $\rightarrow$ nächste Folie
\end{itemize}
\end{block}
\end{frame}

\begin{frame}{Alternierende Büchi-Automaten zu Büchi-Automaten}
\begin{block}{Beweis (Transitionsfunktion Teil 1)}
Für die Transitionsfunktion gilt:
\begin{itemize}
\item für $U\neq\emptyset$:
\begin{align*}
\rho_n((U,V),a)=\{(U',V')| \text{es gibt} X,Y\subseteq S\text{ sodass}&\\
X \text{ erfüllt} \bigwedge_{t \in U}\rho(t,a),\\
Y \text{ erfüllt} \bigwedge_{t_\in V}\rho(t,a),\\
U'=X-F, \text{ und } V'=Y\cup(X\cap F)\},
\end{align*}
\end{itemize}
\end{block}
\end{frame}

\begin{frame}{Alternierende Büchi-Automaten zu Büchi-Automaten}
\begin{block}{Beweis (Transitionsfunktion Teil 2)}
Für die Transitionsfunktion gilt:
\begin{itemize}
\item für $U=\emptyset$:
\begin{align*}
\rho_n((\emptyset,V),a)=\{(U',V')|\text{ es gibt } X,Y\subseteq S \text{ sodass }\\
Y \text{ erfüllt } \bigwedge_{t_\in V}\rho(t,a),\\
U'=Y-F\text{, und } V'=Y\cap F\}.
\end{align*}
\end{itemize}
\end{block}
\end{frame}

\begin{frame}{LTL und Büchi-Automaten}
\begin{block}{Korollar \cite{vardi+96,vardi+94}}
Gegeben eine LTL Formel $\varphi$ ist es möglich einen Büchi-Automaten $A_{\varphi}=(\Sigma,S, S^0,\rho,F)$ zu konstruieren, wobei $\Sigma=2^{Prop}$ und $|S|$ in $2^{O(|\varphi|)}$, sodass $L_{\omega}(A_n)$ genau die Menge der Berechnungen, die die Formel $\varphi$ erfüllen.
\end{block}
\end{frame}
\section{Alternierende Büchi Automaten und LTL}

\subsection{Zielsetzung}
\begin{frame}{\insertsubsection}
\begin{block}{Ziel}
    Fragestellungen des LTL Modelcheckings mit automatentheoretischen Mitteln beantworten.
\end{block}
\vspace*{1.5em}
\begin{block}{Satz: LTL Formel $\rightarrow$ alternierende Büchi-Automaten \cite{vardi+96}}
    Gegeben eine LTL Formel $\varphi$ ist es möglich, einen alternierenden Büchi-Automaten $A_\varphi = (\Sigma, S, s^0, \rho, F)$ zu konstruieren, wobei $\Sigma = 2^{Prop}$ und $|S| = \mathcal{O}(|\varphi|)$, sodass gilt: $L_\omega(A_\varphi)$ bildet exakt die Menge der Berechnungen, die die LTL Formel $\varphi$ erfüllen.
\end{block}
\end{frame}

\subsection{Verfahren nach Vardi \cite{vardi+96}}
\begin{frame}{\insertsubsection}
Um LTL Formeln gut auf alternierende Büchi Automaten übertragen zu können, benötigen wir...
\vspace*{1em}
\begin{enumerate}
    \setlength\itemsep{1.5em}
    \item alternative Definition von Dualismus
    \item Konstruktionsregeln für Automaten
    \item Definition der Übergangsfunktion
\end{enumerate}
\end{frame}

\begin{frame}{\insertsubsection}
\begin{block}{Alternative Definition von Dualismus}
    \vspace*{-1em}
    \begin{equation*}
    \label{aba-zu-ltl:dualismus}
    \begin{split}
        \overline{\varphi} &= \lnot\varphi, \varphi \in S\\
        \overline{\top} &= \bot\\
        \overline{\bot} &= \top\\
        \overline{(\alpha \land \beta)} &= (\overline{\alpha} \lor \overline{\beta})\\
        \overline{(\alpha \lor \beta)} &= (\overline{\alpha} \land \overline{\beta})
    \end{split}
    \end{equation*}
\end{block}
\end{frame}

\begin{frame}{\insertsubsection}
\begin{block}{Konstruktionsregeln für Automaten}
Alternierender Büchi-Automat $A_\varphi = (\Sigma, S, s^0, \rho, F)$:
\vspace*{1em}
    \begin{itemize}
        \setlength\itemsep{1em}
        \item $S$: Menge aller Teilformeln von $\varphi$ inkl. Negation
        \item doppelte Negation vereinfachen: $\lnot\lnot\psi = \psi$
        \item $s^0 = \varphi$
        \item $F$: Menge aller Teilformeln aus $S$, in der Form $\lnot(\sigma U\psi)$
        \item Übergangsfunktion $\rho$
    \end{itemize}
\end{block}
\end{frame}

\begin{frame}{\insertsubsection}
\begin{block}{Die Übergangsfunktion $\rho$}
    \vspace*{-1em}
    \begin{equation*}
    \label{aba-zu-ltl:transitionsfunktion}
    \begin{split}
        \rho(p, a) &= \top, falls \ p \in a,\\
        \rho(p, a) &= \bot, falls \ p \not\in a,\\
        \rho(\sigma \land \psi, a) &= \rho(\sigma, a) \land \rho(\psi, a),\\
        \rho(\sigma \lor \psi, a) &= \rho(\sigma, a) \lor \rho(\psi, a),\\
        \rho(\lnot\psi, a) &= \overline{\rho(\psi, a)},\\
        \rho(X\psi, a) &= \psi,\\
        \rho(\sigma U\psi, a) &= \rho(\psi, a) \lor (\rho(\sigma, a) \land \sigma U\psi)
    \end{split}
    \end{equation*}
\end{block}
\end{frame}

\subsection{Beweis}
\begin{frame}{\insertsubsection}
\begin{itemize}
    \item Laufbaum des alternierenden Büchi-Automaten betrachten
    \item Knoten im Baum beschriftet mit Elementen $\in 2^{Prop} = \Sigma$
    \item zwei Arten von unendlichen Ästen
    \begin{enumerate}
        \item beschriftet mit $(\sigma U\psi)$
        \item beschriftet mit $\lnot(\sigma U\psi)$
    \end{enumerate}
    \item $(\sigma U\psi)$: es muss zwar irgendwann $\psi$ gelten, aber das kann unendlich lange dauern!
    \item $\lnot(\sigma U\psi)$: es muss ab einem bestimmtem Zeitpunkt $\lnot\psi$ gelten -- via Definition der Übergangsfunktion $\rho$
\end{itemize}
\end{frame}

\begin{frame}{\insertsubsection}
    \begin{block}{Unendliche Äste beschriftet mit $\lnot(\sigma U\psi)$}
        \begin{equation*}
        \begin{split}
            \rho(\lnot(\sigma U\psi), a) &= \overline{\rho(\psi, a) \lor (\rho(\sigma, a) \land \sigma U\psi)}\\
            &= \overline{\rho(\psi, a)} \land \overline{(\rho(\sigma, a) \land \sigma U\psi)}\\
            &= \overline{\rho(\psi, a)} \land (\overline{\rho(\sigma, a)} \lor \overline{\sigma U\psi)}\\
            &= \overline{\rho(\psi, a)} \land (\overline{\rho(\sigma, a)} \lor \lnot(\sigma U\psi))\\
        \end{split}
        \end{equation*}
    \end{block}

    \begin{itemize}
    \setlength\itemsep{1em}
        \item Es gilt also $\lnot\psi$ ab dem Zeitpunkt, an dem ein unendlicher Ast mit $\lnot(\sigma U\psi)$ beschriftet ist -- und somit gilt auch $\lnot(\sigma U\psi)$.
        \item Vergleiche die Definition von $F$ 
    \end{itemize}
\end{frame}

\subsection{Zusammenfassung}
\begin{frame}{\insertsubsection}
\begin{block}{Verfahren Überblick}
    \begin{itemize}
        \setlength\itemsep{0.7em}
        \item wir kennen nun Konstruktionsregeln für alternierenden Büchi-Automaten
        \item $L_\omega(A_\varphi):$ Menge aller Berechnungen, die $\varphi$ erfüllen
        \item \textit{Mapping} von LTL Formel zu Automat
    \end{itemize}
\end{block}
\begin{block}{Was nun?}
    \begin{itemize}
        \setlength\itemsep{0.7em}
        \item Eigenschaften von LTL Formeln via automatentheoretische Mittel
        \item Dazu: \textit{normale} Büchi-Automaten verwenden
    \end{itemize}
\end{block}
\end{frame}
\section{Einleitung}

Um die Korrektheit eines Programms zu verifizieren, werden unter anderem sogenannte Modelchecking Verfahren verwendet. Dabei ist die Idee, zunächst ein Modell des zu verifizierenden Programms zu erstellen. Anschließend wird eine Spezifikation angegeben, die das Programm zu erfüllen hat. Der Test, ob ein Programm bzw. dessen Modell eine gegebene Spezifikation erfüllt wird als Modelchecking bezeichnet.

Die Angabe der Programm-Spezifikation erfolgt in einer \textit{formalen Spezifikationssprache}. Weithin bekannt sind \textit{LTL -- Linear-Time Temporal Logic}, \textit{CTL -- Computational Tree Logic} und \textit{CTL* -- Vereinigung aus LTL und CTL}. Im Folgenden beschränken wir uns auf die Betrachtung von LTL. Ziel dieser Arbeit ist -- eng angelehnt an \cite{vardi+96} -- eine Möglichkeit aufzuzeigen, wie verschiedene Fragestellungen des LTL-Modelcheckings mit automatentheoretischen Methoden gelöst werden können. 

Wir werden zunächst Omega Automaten, insbesondere \textit{Büchi} Automaten sowie die Grundeigenschaften von LTL Formeln vorstellen. Der Hauptteil der Arbeit umfasst das Verständnis und die Aufbereitung der Beweise aus \cite{vardi+96}, wobei zunächst alternierende Büchi Automaten und später normale Büchi Automaten mit LTL Formeln in Verbindung gebracht werden.

Abschließend geben wir einen Ausblick auf Modelchecking. Dabei werden wir kurz die Vorteile diskutieren, die sich aus der automatentheoretischen Betrachtung von LTL Formeln ergeben haben.
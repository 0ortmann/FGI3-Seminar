% Endliche Automaten

\begin{frame}{Endliche Automaten}
\begin{block}{Definition Endlicher Automat}
Ein endlicher Automat A ist ein Tupel ($\Sigma, S, S^0, \rho, F$) mit
\vspace{1em}
\begin{itemize}
\setlength\itemsep{1em}
\item $\Sigma$ ist nicht leeres Alphabet,
\item $S$ ist endliche nicht leere Menge von Zuständen,
\item $S^0\subseteq S$ ist nicht leere Menge von Startzuständen,
\item $F\subseteq S$ ist Menge von Endzuständen und
\item $\rho : S \times \Sigma \rightarrow 2^S$ Transitionsfunktion.
\end{itemize}
\end{block}
\end{frame}

\begin{frame}{Endlicher Automat}
\begin{block}{Definition Endlicher Automat (Fortsetzung)}
\begin{itemize}
\setlength\itemsep{1em}
\item Ist $|S^0|=1$ und $|\rho(s,a)|\leq 1$ für alle $s\in S$ und $a\in \Sigma$, dann ist der Automat deterministisch.
\pause
\item Ein Lauf $r$ auf dem Wort $w=a_0,...,a_{n-1}\in \Sigma^\ast$ ist eine Sequenz von Zuständen $s_0,...,s_n$, sodass $s_0\in S^0$ und $s_{i+1}\in \rho(s_i,a_i)$ für $0\leq i < n$ gilt.
\item Ein Lauf $r$ ist akzeptierend, wenn $s_n\in F$ gilt.
\end{itemize}
\end{block}
\end{frame}

% Alternierende endliche Automaten

\begin{frame}{Alternierende Endliche Automaten}
\begin{block}{Alternation}
\begin{itemize}[<+->]
\setlength\itemsep{1em}
\item Nichtdeterminismus ermöglicht existenzielle Entscheidung
\item Duales Konzept ist universelle Wahl
\item $\rightarrow$ Alternation verbindet beide Konzepte
\end{itemize}
\end{block}
\end{frame}

\begin{frame}{Alternierende Endliche Automaten}
\begin{block}{Die Menge $\mathcal{B}^+(X)$}
\begin{itemize}
\setlength\itemsep{1em}
\item $\mathcal{B}^+(X)$ ist die Menge der positiven boolschen Formeln über der Menge X.
\item Beinhaltet auch die Formeln $\top$ (true) und $\bot$ (false).
\pause
\item $Y \subseteq X$ erfüllt eine Formel $\theta \in \mathcal{B}^+(X)$, wenn die Wahrheitsbelegung, die $true$ für die Elemente von $Y$ und $false$ für die Elemente von $X-Y$ zuordnet, $\theta$ erfüllt.
\pause
\item Beispiel \cite{vardi+96}: $Y=\{s_1,s_3\}$ erfüllt die Formel $\theta = (s_1 \vee s_2)\wedge (s_3\vee s_4)$\\
\end{itemize}
\end{block}
\end{frame}

\begin{frame}{Alternierende Endliche Automaten}
\begin{block}{Definition Alternierender Endlicher Automat}
\begin{itemize}
\setlength\itemsep{1em}
\item Unterscheidet sich nur in der Transitionsfunktion von einem nichtdeterministischen Automaten
\item Formal: $\rho: S\times \Sigma \rightarrow \mathcal{B}^+(S)$
\end{itemize}
\end{block}
\end{frame}

\begin{frame}{Alternierende Endliche Automaten}
\begin{block}{Definition Alternierender Endlicher Automat (Fortsetzung)}
Ein Lauf r auf einem $w=a_0,a_1,...,a_{n-1}$ ist ein (unendlicher) S-beschrifteter Baum r mit
\begin{itemize}
\setlength\itemsep{1em}
\item $r(\epsilon)=s^0$,
\pause
\item und für die folgenden Knoten gilt: wenn $|x|=i, r(x)=s$ und $\rho(s,a_i)=\theta$, dann hat $x$ $k$ Kinder $x_1,...,x_k$ für ein $k \leq |S|$ und $r(x_1),...,r(x_k)$ erfüllt $\theta$.
\end{itemize}
\pause
\vspace{1em}
Ein Lauf r wird akzeptiert, wenn alle Knoten in der Tiefe $n$ mit Zuständen aus $F$ markiert sind.
\end{block}
\end{frame}

% Büchi-Automaten

\begin{frame}{Büchi-Automaten}
\begin{block}{Definition Büchi-Automat}
\begin{itemize}
\setlength\itemsep{1em}
\item Büchi-Automaten akzeptieren unendliche Wörter $a_1,a_2,... \in \Sigma^\omega$
\item Definition wie endlicher Automat, aber besitzen nur einen Startzustand $s^0$
\pause
\item Ein Lauf $r$ auf dem Wort $w=a_0,a_1,...\in \Sigma^\omega$ ist eine unendliche Folge von Zuständen $s_0,s_1,...$, wobei $s_0\in S^0$ und $s_{i+1}\in \rho(s_i,a_i)$ für alle $i\geq0$ gilt.
\pause
\item $lim(r)$ ist Menge \{$s|s=s_i$ für unendlich viele i's\}
\item Ein Lauf $r$ ist akzeptierend, wenn $lim(r)\cap F \neq \emptyset$ gilt.
\end{itemize}
%Ein Büchi-Automat A ist ein Tupel ($\Sigma, S, s^0, \rho, F$) mit
%\begin{itemize}
%\item $\Sigma$ ist nicht leeres Alphabet,
%\item $S$ ist endliche nicht leere Menge von Zuständen,
%\item $s^0$ ist ein Startzustand,
%\item $F\subseteq S$ ist Menge von Endzuständen und
%\item $\rho : S \times \Sigma \rightarrow 2^S$ Transitionsfunktion.
%\end{itemize}
\end{block}
\end{frame}

\begin{frame}{Alternierende Büchi-Automaten}
\begin{block}{Definition Alternierender Büchi-Automat}
\begin{itemize}
\setlength\itemsep{1em}
\item Unterschied zu Büchi-Automaten liegt nur in der Definition der Transaktionsfunktion
\item Formal: $\rho: S\times \Sigma \rightarrow \mathcal{B}^+(S)$
\end{itemize}
%Ein Büchi-Automat A ist ein Tupel ($\Sigma, S, s^0, \rho, F$) mit
%\begin{itemize}
%\item $\Sigma$ ist nicht leeres Alphabet,
%\item $S$ ist endliche nicht leere Menge von Zuständen,
%\item $s^0$ ist ein Startzustand,
%\item $F\subseteq S$ ist Menge von Endzuständen und
%\item \textbf{$\rho : S \times \Sigma \rightarrow \mathcal{B}^+(S)$} Transitionsfunktion.
%\end{itemize}
\end{block}
\end{frame}

\begin{frame}{Alternierende Büchi-Automaten}
\begin{block}{Definition Alternierender Büchi-Automat (Fortsetzung)}
Ein Lauf r auf einem unendlichen Wort $w=a_0,a_1,...$ ist ein (unendlicher) S-beschrifteter Baum r mit
\begin{itemize}
\setlength\itemsep{1em}
\item $r(\epsilon)=s^0$,
\pause
\item und für die folgenden Knoten gilt: wenn $|x|=i, r(x)=s$ und $\rho(s,a_i)=\theta$, dann hat $x$ $k$ Kinder $x_1,...,x_k$ für ein $k \leq |S|$ und $r(x_1),...,r(x_k)$ erfüllt $\theta$.
\end{itemize}
\pause
\vspace{1em}
Ein Lauf r wird akzeptiert, wenn jeder unendliche Zweig unendlich viele Markierungen aus der Endzustandsmenge enthält.
\end{block}
\end{frame}


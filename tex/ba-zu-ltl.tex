\section{LTL zu Büchi-Automaten}

In diesem Abschnitt wird zuerst der Zusammenhang von alternierenden Büchi-Automaten und einfachen Büchi Automaten beschrieben. Daraus wird die Möglichkeit, LTL-Formeln auch mit einfachen Büch-Automaten auf Erfüllbarkeit zu testen, abgeleitet.

\subsection{Alternierende Büchi-Automaten zu Büchi-Automaten}

Alternierende Büchi-Automaten sind nicht ausdrucksstärker als nichtdetermistische Büchi-Automaten. Es gilt folgender Satz für die Beziehung zwischen ihnen:
\begin{satz}\cite{vardi+96,miyano+84}
\label{ABA>BA}
Sei A ein alternierender Büchi-Automat. Dann gibt es einen nicht nichtdeterministischen Büchi-Automaten $A_n$ sodass $L_{\omega}(A_n)=L_{\omega}(A)$ gilt.
\end{satz}
\textbf{Beweis:} Der Automat $A_n$ nimmt einen Lauf von A an und merkt sich zu jedem Zeitpunkt die gesamte Ebene eines Laufes von A. Dabei werden schrittweise die Eingangssymbole gelesen und der Lauf jeweils erraten. Aus der Definition eines Laufs eines alternierenden Büchi-Automaten in Abschnitt \ref{subsec:aba} ergibt sich, das die einzelnen Ebenen Bäume sind, die sich $A_n$ speichern muss. Der Automat $A_n$ merkt sich dabei das Auftreten von akzeptierenden Zuständen um entscheiden zu können, ob ein unendlicher Pfad unendlich oft einen Endzustand durchläuft oder nicht \cite{vardi+96}. Dafür nutzt der Automat eine Aufteilung der einzelnen Ebenen des Laufes in zwei Mengen, wobei eine Menge die Zweige enthält, die vor kurzem einen Endzustand getroffen haben und die andere Menge die Zweige enthält, die vor kurzem keinen Endzustand erreicht haben (vgl. \cite{vardi+96}).

Die genaue Konstruktion, wie sie in \cite{vardi+96} genutzt wird, ist im Folgenden beschrieben. Gegeben sei der Automat $A=(\Sigma,S, s^0, \rho, F)$. Dann sieht der Automat $A_n$, der die gleiche Sprache akzeptiert, wie folgt aus: $A_n=(\Sigma,S_n,S^0,\rho_n,F_n)$, wobei $S_n=2^S\times2^S$, $F_n={\emptyset}\times2^S$ und für die Transitionsfunktion gilt \begin{itemize}
\item für $U\neq\emptyset$:
\begin{align*}
\rho_n((U,V),a)=\{(U',V')| \text{ es gibt } X,Y\subseteq S\text{ sodass}&\\
X \text{ erfüllt} \bigwedge_{t \in U}\rho(t,a),\\
Y \text{ erfüllt} \bigwedge_{t_\in V}\rho(t,a),\\
U'=X-F, \text{ und } V'=Y\cup(X\cap F)\},
\end{align*}
\item für $U=\emptyset$:
\begin{align*}
\rho_n((\emptyset,V),a)=\{(U',V')|\text{ es gibt } X,Y\subseteq S \text{ sodass }\\
Y \text{ erfüllt } \bigwedge_{t_\in V}\rho(t,a),\\
U'=Y-F\text{, und } V'=Y\cap F\}.
\end{align*}
\end{itemize}
Der Beweis der Konstruktion erfordert eine sorgfältige Analyse der akzeptierenden Läufe von $A$ \cite{vardi+96}.
$\square$



\subsection{LTL zu Büchi-Automaten}

Unter der Anwendung von Satz \ref{ABA>BA} aus dem vorigen Abschnitt kann folgende Schlussfolgerung gezogen werden:

\begin{korol}\cite{vardi+96,vardi+94}
Gegeben eine LTL Formel $\varphi$ ist es möglich einen Büchi-Automaten $A_{\varphi}=(\Sigma,S, S^0,\rho,F)$ zu konstruieren, wobei $\Sigma=2^{Prop}$ und $|S|$ in $2^{O(|\varphi|)}$, sodass $L_{\omega}(A_n)$ genau die Menge der Berechnungen, die die Formel $\varphi$ erfüllen.
\end{korol}

Der Beweis dieser Schlussfolgerung teilt sich in den Beweis der Logik und der Kombinatorik. Die Logik wurde in Satz \ref{ABA>BA} behandelt und die Kombinatorik ist durch Satz \ref{LTL->ABA} erläutert worden.


%Teil von Jennifer: Beziehung von alternierenden Büchi Automaten zu normalen Büchi Automaten. Daraus folgerbar auch LTL mit normalen Büchi Automaten.
\section{LTL zu Büchi-Automaten}

In diesem Abschnitt wird der Zusammenhang von alternierenden Büchi-Automaten und einfachen Büchi Automaten beschrieben. Daraus wird die Möglichkeit, LTL-Formeln auch mit einfachen Büch- Automaten auf Erfüllbarkeit zu testen, abgeleitet.

\subsection{Alternierende Büchi-Automaten zu Büchi-Automaten}

Es gilt folgender Satz für die Beziehung zwischen einem alternierenden Büchi-Automaten und einem nichtdeterministischen Büchi-Automaten:
\begin{satz}\cite{vardi+96,miyano+84}
\label{ABA>BA}
Sei A ein alternierender Büchi-Automat. Dann gibt es einen nicht nichtdeterministischen Büchi-Automaten $A_n$ sodass $L_{\omega}(A_n)=L_{\omega}(A)$ gilt.
\end{satz}
\textbf{Beweis:}

\subsection{LTL zu Büchi-Automaten}

Unter der Anwendung von dem Satz \ref{ABA>BA} aus dem vorigen Abschnitt kann folgende Schlussfolgerung gezogen werden:

\begin{korol}\cite{vardi+96,vardi+94}
Gegeben eine LTL Formel $\varphi$ ist es möglich einen Büchi-Automaten $A_{\varphi}=(\Sigma,S, S^0,\rho,F)$ zu konstruieren, wobei $\Sigma=2^{Prop}$ und $|S|$ in $2^{O(|\varphi|)}$, sodass $L_{\omega}(A_n)$ genau die Menge der Berechnungen, die die Formel $\varphi$ erfüllen.
\end{korol}

%Teil von Jennifer: Beziehung von alternierenden Büchi Automaten zu normalen Büchi Automaten. Daraus folgerbar auch LTL mit normalen Büchi Automaten.
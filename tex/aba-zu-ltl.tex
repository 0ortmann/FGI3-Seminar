\section{LTL zu alternierenden Büchi-Automaten}

Der folgende Abschnitt stellt ein Übersetzungsverfahren von LTL Formeln zu alternierenden Büchi-Automaten vor. Darauf aufbauend kann die Erfüllbarkeit einer LTL Formel mit Hilfe des äquivalenten alterniereden Büchi-Automaten festgestellt werden. Das Verfahren und der Beweis nach \cite{vardi+96} werden diskutiert.


\subsection{Verfahren}
\label{subsection:verfahren}
Im vorangegangenen Abschnitt wurde die Semantik einer LTL Formel anhand einer Berechnung definiert. Berechnungen können auch als unendliche Worte über dem Alphabet $2^{Prop}$ dargestellt werden\cite{vardi+96}. Aufgrund dieser Verbindung gilt:
\begin{satz}\cite{vardi+96}
\label{LTL->ABA}
    Gegeben eine LTL Formel $\varphi$ ist es möglich, einen alternierenden Büchi-Automaten $A_\varphi = (\Sigma, S, s^0, \rho, F)$ zu konstruieren, wobei $\Sigma = 2^{Prop}$ und $|S| = \mathcal{O}(|\varphi|)$, sodass gilt: $L_\omega(A_\varphi)$ bildet exakt die Menge der Berechnungen, die die LTL Formel $\varphi$ erfüllen.
\end{satz}
Für die gleich folgende Definition des 5-Tupels des alternierenden Büchi-Automaten benötigen wir eine leicht veränderte Definition für Dualismus; analog zu \cite{vardi+96} bilden wir $\overline{\varphi}$ durch Vertauschen von $\land$ und $\lor$, das Vertauschen von $\top$ und $\bot$ und durch das Negieren von Teilformen in $S$:
\begin{equation}
\label{aba-zu-ltl:dualismus}
\begin{split}
    \overline{\varphi} &= \lnot\varphi, \varphi \in S\\
    \overline{\top} &= \bot\\
    \overline{\bot} &= \top\\
    \overline{(\alpha \land \beta)} &= (\overline{\alpha} \lor \overline{\beta})\\
    \overline{(\alpha \lor \beta)} &= (\overline{\alpha} \land \overline{\beta})
\end{split}
\end{equation}
Mit dieser Definition von Dualismus können wir nun die Konstruktionsregeln für den alternierenden Büchi-Automaten $A_\varphi$ nach \cite{vardi+96} angeben: Die Gesamtzustandsmenge $S$ ist die Menge aller Teilformeln von $\varphi$ und deren Negationen. Dabei vereinfachen wir doppelte Negation, sodass $\lnot\lnot\psi = \psi$. Der Startzustand $s^0$ ist $\varphi$. 
Die Menge der Endzustände $F$ wird definiert als all jene Teilformeln aus $S$, die in der Form $\lnot(\sigma U\psi)$ sind. Abschließend wird die Übergangsfunktion $\rho$ wie folgt definiert:
\begin{equation}
\label{aba-zu-ltl:transitionsfunktion}
\begin{split}
    \rho(p, a) &= \top, falls \ p \in a,\\
    \rho(p, a) &= \bot, falls \ p \not\in a,\\
    \rho(\sigma \land \psi, a) &= \rho(\sigma, a) \land \rho(\psi, a),\\
    \rho(\sigma \lor \psi, a) &= \rho(\sigma, a) \lor \rho(\psi, a),\\
    \rho(\lnot\psi, a) &= \overline{\rho(\psi, a)},\\
    \rho(X\psi, a) &= \psi,\\
    \rho(\sigma U\psi, a) &= \rho(\psi, a) \lor (\rho(\sigma, a) \land \sigma U\psi)
\end{split}
\end{equation}


\subsection{Beweis}

Wie in Abschnitt \ref{subsec:aba} definiert, ist ein Lauf eines alternierenden Büchi-Automaten als unendlicher, beschrifteter Baum zu betrachten. Die möglichen Beschriftungen der Knoten des Baumes sind alle Elemente der Zustandsmenge des Automaten. Sei nun der alternierende Büchi-Automat $A_\varphi$ nach dem Verfahren aus \ref{subsection:verfahren} konstruiert. Die möglichen Beschriftungen des Lauf-Baumes sind genau all jene Teilformeln der LTL Formel $\varphi$ und deren Negationen.

Sei nun $l$ ein Lauf von $A_\varphi$. Der Laufbaum $l$ kann nur zwei Arten von unendlichen Ästen haben -- entweder ist ein unendlicher Ast beschriftet mit $\sigma U\psi$ oder mit $\lnot(\sigma U\psi)$ \cite{vardi+96}. Diese beiden möglichen Beschriftungen bringen unterschiedliche Eigenschaften. 

Wenn ein unendlicher Ast ab einem bestimmten Punkt mit $\sigma U\psi$ beschriftet ist, so muss zwar irgendwann $\psi$ gelten, aber das kann unendlich lange dauern; durch die Definition von $\sigma U\psi$ kann nicht mit Bestimmtheit vorhergesagt werden, zu welchem Zeitpunkt $\psi$ eintritt. Allerdings kann für einen unendlichen Ast, der ab einem bestimmten Punkt mit $\lnot(\sigma U\psi)$ beschriftetet ist, mit Bestimmtheit festgestellt werden, dass $\sigma U\psi$ tatsächlich nicht gilt -- über die Definition von $\rho$:
\begin{equation}
\begin{split}
    \rho(\lnot(\sigma U\psi), a) &= \overline{\rho(\psi, a) \lor (\rho(\sigma, a) \land \sigma U\psi)}\\
    &= \overline{\rho(\psi, a)} \land \overline{(\rho(\sigma, a) \land \sigma U\psi)}\\
    &= \overline{\rho(\psi, a)} \land (\overline{\rho(\sigma, a)} \lor \overline{\sigma U\psi)}\\
    &= \overline{\rho(\psi, a)} \land (\overline{\rho(\sigma, a)} \lor \lnot(\sigma U\psi))\\
\end{split}
\end{equation}
Es wird also ab einem bestimmten Punkt $\lnot\psi$ gelten und somit $\lnot(\sigma U\psi)$ (vergleiche \cite{vardi+96}).

In der Verfahrensbeschreibung in \ref{subsection:verfahren} wurde die Endzustandsmenge definiert als "`all jene Teilformeln aus $S$, die in der Form $\lnot(\sigma U\psi)$ sind"'. $\square$

Abschließend lässt sich festhalten, dass LTL Formeln sich auf alternierende Büchi-Automaten übertragen lassen. Wird ein Büchi-Automat nach obigen Verfahren für eine Formel $\varphi$ konstruiert, so ist die akzeptierende Sprache des Automaten exakt die Menge der Berechnungen, die $\varphi$ erfüllen.
\section{Ausblick: Modelchecking}

\subsection{Überblick}
\begin{frame}{\insertsubsection}
    \begin{block}{Programmmodelle}
        \begin{itemize}
            \item gelabeltes Transitionssystem
            \item \textit{finite state} Programm (x)
        \end{itemize}
    \end{block}
    \begin{block}{Finite state Programm}
    $P = (W, w_0, R, V)$
        \begin{itemize}
            \item $W$: Menge aller Zustände
            \item $w_0$: Startzustand
            \item $R \subseteq W^2$: totale Erreichbarkeitsrelation
            \item $V$: Funktion, die Wahrheitswerte zuweist; $V: W \rightarrow 2^{Prop}$
        \end{itemize}
    \end{block}
\end{frame}

\subsection{Finite State Programme}
\begin{frame}{\insertsubsection}
    \begin{block}{Berechnung in einem finite state Programm \cite{vardi+96}}
    \begin{itemize}
        \item unendliche Zustandsfolge: $u = u_0, u_1, ... $ wobei $\forall i \geq 0: u_i \in W$
        \item $u_0 = w_0$
        \item weise Zuständen in $u$ Aussagen zu
        \item $\pi = V(u_0), V(u_1), ...$
    \end{itemize}
    \end{block}
    \begin{block}{Erfüllbarkeit}
        Wir sagen, dass $P$ eine gegebene LTL Formel $\varphi$ erfüllt, wenn \textit{alle} Berechnungen in $P$ die Formel $\varphi$ erfüllen.
    \end{block}
\end{frame}

\subsection{Verifikationsproblem}
\begin{frame}{\insertsubsection}
    Die Entscheidung, ob ein finite state Programm $P$ eine Formel $\varphi$ erfüllt wird als {Verifikationsproblem} bezeichnet.
    \begin{block}{Konstruktion}
        Entscheidung mit automatentheoretischen Mitteln (siehe \cite[kap. 4.2]{vardi+96})
        \begin{itemize}
            \item Überführung finite state Programm $\rightarrow$ Büchi-Automat
            \item jeder Lauf akzeptierend
            \item $L_\omega(A_P)$: Menge aller Berechnungen in $P$
        \end{itemize} 
    \end{block}
    \begin{block}{Reduktion des Problems}
        Verifikationsproblem Programm $P$ und Formel $\varphi$ wird reduziert:
        erfüllen alle akzeptierenden Worte von $A_P$ die Formeln $\varphi$?
    \end{block}
\end{frame}
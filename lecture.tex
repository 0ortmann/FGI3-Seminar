% !TeX program = xelatex
% !TeX spellcheck = de_DE
\documentclass{beamer}
\usepackage{fontspec}
\usepackage{polyglossia}
\usepackage{hyperref}
\usepackage{xspace}
\usepackage{array}
\usepackage{fontawesome}
\usepackage[normalem]{ulem}

\setsansfont{Calibri}
\setdefaultlanguage{german}
\usetheme{UHH}

\def\authors#1#2{\author[#1]{#2}}
\def\email#1{\texttt{#1}}
\title{Omega Automaten und LTL}
\subtitle{Seminararbeit zu der Veranstaltung \\Formale Grundlagen der Informatik 3}
\authors{Jennifer~Soltau \& Felix~Ortmann}{Jennifer~Soltau und Felix~Ortmann\\ \email{\{2soltau,0ortmann\}@informatik.uni-hamburg.de}}
\institute{Universität Hamburg\\Fachbereich Informatik\\Vogt-Kölln-Straße 30\\22527 Hamburg}


\date{22-26. Februar 2016}
\titlegraphic{\includegraphics[width=3cm]{img/uhh.pdf}}

\setcounter{tocdepth}{1}
\def\tocname{Gliederung des Vortrags}
\AtBeginSection{\frame{\frametitle{\tocname}\tableofcontents[currentsection]}}

\newcolumntype{x}[1]{>{\centering\arraybackslash}m{#1}}
\setbeamertemplate{headline}[default]

\defbeamertemplate*{title page}{customized}[1][]
{
	{\color{leuchtrot}
	\usebeamerfont{title}\centering\inserttitle\par
	\centering\usebeamerfont{subtitle}\insertsubtitle\par}
	\bigskip
	\centering\usebeamerfont{author}\insertauthor\par
	\bigskip
	\usebeamerfont{institute}\insertinstitute\par
	\bigskip
	\usebeamerfont{date}\insertdate\par
}

\begin{document}

\fontsize{14pt}{14pt}
	
% Titelfolie
{\setbeamercolor{title page}{bg=leuchtrot}
\begin{frame}%
	\titlepage
\end{frame}}

% Gliederung
\begin{frame}{\tocname}
	\tableofcontents
\end{frame}

\section{Einführung}
\section{Omega Automaten}
\section{LTL}
\section{Alternierende Büchi Automaten und LTL}
\section{LTL auf normalen Büchi Automaten}
\section{Ausblick: Modelchecking}
\section{Zusammenfassung}

% Literatur
\section{\bibname}
\begin{frame}[allowframebreaks]{\bibname}
	\AtBeginSection{}
	\nocite{*}
	\bibliographystyle{abbrv}
	\bibliography{bib}
\end{frame}
\begin{frame}[allowframebreaks]{Bildquellen}
\end{frame}


\end{document}